
\documentclass[10pt,twocolumn,letterpaper]{article}

\usepackage{cvpr}
\usepackage{times}
\usepackage{epsfig}
\usepackage{graphicx}
\usepackage{amsmath}
\usepackage{amssymb}


% Include other packages here, before hyperref.

% If you comment hyperref and then uncomment it, you should delete
% egpaper.aux before re-running latex.  (Or just hit 'q' on the first latex
% run, let it finish, and you should be clear).
\usepackage[pagebackref=true,breaklinks=true,letterpaper=true,colorlinks,bookmarks=false]{hyperref}

 \cvprfinalcopy % *** Uncomment this line for the final submission

\def\cvprPaperID{****} % *** Enter the CVPR Paper ID here
\def\httilde{\mbox{\tt\raisebox{-.5ex}{\symbol{126}}}}

% Pages are numbered in submission mode, and unnumbered in camera-ready
\ifcvprfinal\pagestyle{empty}\fi
\begin{document}

%%%%%%%%% TITLE
\title{Cheating Rock Scissor Paper}

\author{Junha Chun\\
Dept. of Electrical and Computer Engineering\\
Seoul National University \\
{\tt\small nikriz@snu.ac.kr}
% For a paper whose authors are all at the same institution,
% omit the following lines up until the closing ``}''.
% Additional authors and addresses can be added with ``\and'',
% just like the second author.
% To save space, use either the email address or home page, not both
\and
Dongsu Zhang\\
Dept. of Computer Science and Engineering\\
Seoul National University\\
{\tt\small 96lives@snu.ac.kr}
}

\maketitle
%\thispagestyle{empty}
%%%%%%%% BODY TEXT
\section{Overview}
Rock Paper Scissors(RPS) is a simple hand game that is played all over the world instantly and with nothing. RPS is highly affected by luck, each player has equal one-third chance of win, draw, or loss according to the ‘throw’(rock, scissors, or paper) the players play. Also, there could be psychological strategies since a human cannot play true randomness. 
\paragraph{}
Our goal of this project is to predict opponent’s throw as fast as the program can in real-time, and win the game with high probability with computer vision techniques. We will first try with rule-based classical computer vision techniques, and then continue with machine learning models.


%% Section 2. Paper Survey
\section{Paper Survey}

%% Section 2.1 Classical vision
\subsection{Classical Computer Vision}
For classical computer vision techniques, open source done by \href{https://github.com/lzane/Fingers-Detection-using-OpenCV-and-Python}{lzane} will be referenced. This open source uses background subtraction and grey scaling to detect the hand and finds contours to detect number of fingers.
Although this model detects hands, it did not work well  due to \textbf{FILL!!!} . Also, this open source ignores the process of fingers moving. To estimate what the RPS player will throw before the pose is definitely recognizable, the moving process is needed as data.

%% Section 2.2 Neural Network
\subsection{Hand Pose Estimation with Neural Network}
This method is available with fully open hands, and not certain if it would work well while opening the hands. \href{https://arxiv.org/pdf/1704.02224.pdf}{Hand3D} and \href{https://arxiv.org/pdf/1611.08050.pdf}{Openpose} uses deep neural networks to detect hand joints. 
These methods detect every finger joints and hand poses but it’s not sure their running time is fast enough to be ran in real-time situations. Especially in openpose, the detection of the finger joint is made after the detection of the human body detection.
\paragraph{}
Both methods are basically image processing, so we should implement them in real time video. Also, in order to predict the opponent’s ‘throw’, modification of open-sourced algorithms and precise rule-based algorithm with finger numbers or joint coordinates are needed.

%% Section Dataset Description
\section{Dataset Description}
There are lots of hand RGBD image datasets open online.
\begin{itemize}
\item Own filmed dataset (video and picture), asking acquaintances for a short photo. 30 people or more are estimated.
\item \href{https://cims.nyu.edu/~tompson/NYU_Hand_Pose_Dataset.htm}{NYU hand dataset} 
\item \href{http://icvl.ee.ic.ac.uk/hands17/challenge/}{ICVL hand dataset} 
\item \href{https://arxiv.org/pdf/1704.02612.pdf}{BigHand 2.2M} 
\end{itemize}

\end{document}
